% Options for packages loaded elsewhere
\PassOptionsToPackage{unicode}{hyperref}
\PassOptionsToPackage{hyphens}{url}
\PassOptionsToPackage{dvipsnames,svgnames,x11names}{xcolor}
%
\documentclass[
  letterpaper,
  DIV=11,
  numbers=noendperiod]{scrartcl}

\usepackage{amsmath,amssymb}
\usepackage{lmodern}
\usepackage{iftex}
\ifPDFTeX
  \usepackage[T1]{fontenc}
  \usepackage[utf8]{inputenc}
  \usepackage{textcomp} % provide euro and other symbols
\else % if luatex or xetex
  \usepackage{unicode-math}
  \defaultfontfeatures{Scale=MatchLowercase}
  \defaultfontfeatures[\rmfamily]{Ligatures=TeX,Scale=1}
\fi
% Use upquote if available, for straight quotes in verbatim environments
\IfFileExists{upquote.sty}{\usepackage{upquote}}{}
\IfFileExists{microtype.sty}{% use microtype if available
  \usepackage[]{microtype}
  \UseMicrotypeSet[protrusion]{basicmath} % disable protrusion for tt fonts
}{}
\makeatletter
\@ifundefined{KOMAClassName}{% if non-KOMA class
  \IfFileExists{parskip.sty}{%
    \usepackage{parskip}
  }{% else
    \setlength{\parindent}{0pt}
    \setlength{\parskip}{6pt plus 2pt minus 1pt}}
}{% if KOMA class
  \KOMAoptions{parskip=half}}
\makeatother
\usepackage{xcolor}
\setlength{\emergencystretch}{3em} % prevent overfull lines
\setcounter{secnumdepth}{5}
% Make \paragraph and \subparagraph free-standing
\ifx\paragraph\undefined\else
  \let\oldparagraph\paragraph
  \renewcommand{\paragraph}[1]{\oldparagraph{#1}\mbox{}}
\fi
\ifx\subparagraph\undefined\else
  \let\oldsubparagraph\subparagraph
  \renewcommand{\subparagraph}[1]{\oldsubparagraph{#1}\mbox{}}
\fi

\usepackage{color}
\usepackage{fancyvrb}
\newcommand{\VerbBar}{|}
\newcommand{\VERB}{\Verb[commandchars=\\\{\}]}
\DefineVerbatimEnvironment{Highlighting}{Verbatim}{commandchars=\\\{\}}
% Add ',fontsize=\small' for more characters per line
\newenvironment{Shaded}{}{}
\newcommand{\AlertTok}[1]{\textcolor[rgb]{1.00,0.00,0.00}{\textbf{#1}}}
\newcommand{\AnnotationTok}[1]{\textcolor[rgb]{0.38,0.63,0.69}{\textbf{\textit{#1}}}}
\newcommand{\AttributeTok}[1]{\textcolor[rgb]{0.49,0.56,0.16}{#1}}
\newcommand{\BaseNTok}[1]{\textcolor[rgb]{0.25,0.63,0.44}{#1}}
\newcommand{\BuiltInTok}[1]{#1}
\newcommand{\CharTok}[1]{\textcolor[rgb]{0.25,0.44,0.63}{#1}}
\newcommand{\CommentTok}[1]{\textcolor[rgb]{0.38,0.63,0.69}{\textit{#1}}}
\newcommand{\CommentVarTok}[1]{\textcolor[rgb]{0.38,0.63,0.69}{\textbf{\textit{#1}}}}
\newcommand{\ConstantTok}[1]{\textcolor[rgb]{0.53,0.00,0.00}{#1}}
\newcommand{\ControlFlowTok}[1]{\textcolor[rgb]{0.00,0.44,0.13}{\textbf{#1}}}
\newcommand{\DataTypeTok}[1]{\textcolor[rgb]{0.56,0.13,0.00}{#1}}
\newcommand{\DecValTok}[1]{\textcolor[rgb]{0.25,0.63,0.44}{#1}}
\newcommand{\DocumentationTok}[1]{\textcolor[rgb]{0.73,0.13,0.13}{\textit{#1}}}
\newcommand{\ErrorTok}[1]{\textcolor[rgb]{1.00,0.00,0.00}{\textbf{#1}}}
\newcommand{\ExtensionTok}[1]{#1}
\newcommand{\FloatTok}[1]{\textcolor[rgb]{0.25,0.63,0.44}{#1}}
\newcommand{\FunctionTok}[1]{\textcolor[rgb]{0.02,0.16,0.49}{#1}}
\newcommand{\ImportTok}[1]{#1}
\newcommand{\InformationTok}[1]{\textcolor[rgb]{0.38,0.63,0.69}{\textbf{\textit{#1}}}}
\newcommand{\KeywordTok}[1]{\textcolor[rgb]{0.00,0.44,0.13}{\textbf{#1}}}
\newcommand{\NormalTok}[1]{#1}
\newcommand{\OperatorTok}[1]{\textcolor[rgb]{0.40,0.40,0.40}{#1}}
\newcommand{\OtherTok}[1]{\textcolor[rgb]{0.00,0.44,0.13}{#1}}
\newcommand{\PreprocessorTok}[1]{\textcolor[rgb]{0.74,0.48,0.00}{#1}}
\newcommand{\RegionMarkerTok}[1]{#1}
\newcommand{\SpecialCharTok}[1]{\textcolor[rgb]{0.25,0.44,0.63}{#1}}
\newcommand{\SpecialStringTok}[1]{\textcolor[rgb]{0.73,0.40,0.53}{#1}}
\newcommand{\StringTok}[1]{\textcolor[rgb]{0.25,0.44,0.63}{#1}}
\newcommand{\VariableTok}[1]{\textcolor[rgb]{0.10,0.09,0.49}{#1}}
\newcommand{\VerbatimStringTok}[1]{\textcolor[rgb]{0.25,0.44,0.63}{#1}}
\newcommand{\WarningTok}[1]{\textcolor[rgb]{0.38,0.63,0.69}{\textbf{\textit{#1}}}}

\providecommand{\tightlist}{%
  \setlength{\itemsep}{0pt}\setlength{\parskip}{0pt}}\usepackage{longtable,booktabs,array}
\usepackage{calc} % for calculating minipage widths
% Correct order of tables after \paragraph or \subparagraph
\usepackage{etoolbox}
\makeatletter
\patchcmd\longtable{\par}{\if@noskipsec\mbox{}\fi\par}{}{}
\makeatother
% Allow footnotes in longtable head/foot
\IfFileExists{footnotehyper.sty}{\usepackage{footnotehyper}}{\usepackage{footnote}}
\makesavenoteenv{longtable}
\usepackage{graphicx}
\makeatletter
\def\maxwidth{\ifdim\Gin@nat@width>\linewidth\linewidth\else\Gin@nat@width\fi}
\def\maxheight{\ifdim\Gin@nat@height>\textheight\textheight\else\Gin@nat@height\fi}
\makeatother
% Scale images if necessary, so that they will not overflow the page
% margins by default, and it is still possible to overwrite the defaults
% using explicit options in \includegraphics[width, height, ...]{}
\setkeys{Gin}{width=\maxwidth,height=\maxheight,keepaspectratio}
% Set default figure placement to htbp
\makeatletter
\def\fps@figure{htbp}
\makeatother

\usepackage{booktabs}
\usepackage{longtable}
\usepackage{array}
\usepackage{multirow}
\usepackage{wrapfig}
\usepackage{float}
\usepackage{colortbl}
\usepackage{pdflscape}
\usepackage{tabu}
\usepackage{threeparttable}
\usepackage{threeparttablex}
\usepackage[normalem]{ulem}
\usepackage{makecell}
\usepackage{xcolor}
\KOMAoption{captions}{tableheading}
\makeatletter
\makeatother
\makeatletter
\makeatother
\makeatletter
\@ifpackageloaded{caption}{}{\usepackage{caption}}
\AtBeginDocument{%
\ifdefined\contentsname
  \renewcommand*\contentsname{Table of contents}
\else
  \newcommand\contentsname{Table of contents}
\fi
\ifdefined\listfigurename
  \renewcommand*\listfigurename{List of Figures}
\else
  \newcommand\listfigurename{List of Figures}
\fi
\ifdefined\listtablename
  \renewcommand*\listtablename{List of Tables}
\else
  \newcommand\listtablename{List of Tables}
\fi
\ifdefined\figurename
  \renewcommand*\figurename{Şekil}
\else
  \newcommand\figurename{Şekil}
\fi
\ifdefined\tablename
  \renewcommand*\tablename{Tablo}
\else
  \newcommand\tablename{Tablo}
\fi
}
\@ifpackageloaded{float}{}{\usepackage{float}}
\floatstyle{ruled}
\@ifundefined{c@chapter}{\newfloat{codelisting}{h}{lop}}{\newfloat{codelisting}{h}{lop}[chapter]}
\floatname{codelisting}{Listing}
\newcommand*\listoflistings{\listof{codelisting}{List of Listings}}
\makeatother
\makeatletter
\@ifpackageloaded{caption}{}{\usepackage{caption}}
\@ifpackageloaded{subcaption}{}{\usepackage{subcaption}}
\makeatother
\makeatletter
\makeatother
\ifLuaTeX
  \usepackage{selnolig}  % disable illegal ligatures
\fi
\IfFileExists{bookmark.sty}{\usepackage{bookmark}}{\usepackage{hyperref}}
\IfFileExists{xurl.sty}{\usepackage{xurl}}{} % add URL line breaks if available
\urlstyle{same} % disable monospaced font for URLs
\hypersetup{
  pdftitle={Krizantem},
  pdfauthor={Bengisu ÖZBİLEN 21255061 Ferhat ARSLAN 21255601},
  colorlinks=true,
  linkcolor={blue},
  filecolor={Maroon},
  citecolor={Blue},
  urlcolor={Blue},
  pdfcreator={LaTeX via pandoc}}

\title{Krizantem}
\author{Bengisu ÖZBİLEN 21255061 Ferhat ARSLAN 21255601}
\date{}

\begin{document}
\maketitle
\renewcommand*\contentsname{Table of contents}
{
\hypersetup{linkcolor=}
\setcounter{tocdepth}{3}
\tableofcontents
}
\hypertarget{verilerin-okutulmasi}{%
\section{VERİLERİN OKUTULMASI}\label{verilerin-okutulmasi}}

\begin{Shaded}
\begin{Highlighting}[]
\FunctionTok{library}\NormalTok{(readxl)}
\NormalTok{Krizantem }\OtherTok{\textless{}{-}} \FunctionTok{read\_excel}\NormalTok{(}\StringTok{"Krizantem.xlsx"}\NormalTok{)}
\FunctionTok{View}\NormalTok{(Krizantem)}
\end{Highlighting}
\end{Shaded}

\hypertarget{projenin-konusu-ve-amaci}{%
\section{PROJENİN KONUSU VE AMACI}\label{projenin-konusu-ve-amaci}}

Veri seti, bir grup denek ve onların uyku düzenleri hakkında bilgi
içerir. Her test deneği benzersiz bir ``Subject ID'' ile tanımlanır ve
yaşları ve cinsiyetleri de kaydedilir. ``Uyku Zamanı'' ve ``Uyanma
zamanı'' özellikleri, her bir öznenin her gün ne zaman yatıp uyandığını
gösterir ve ``Uyku süresi'' özelliği, her öznenin saat olarak uyuduğu
toplam süreyi kaydeder. ``Uyku verimliliği'' özelliği, yatakta geçirilen
sürenin gerçekte uykuda geçirilen süreye oranının bir ölçüsüdür. ``REM
uyku yüzdesi'', ``Derin uyku yüzdesi'' ve ``Hafif uyku yüzdesi''
özellikleri, her öznenin uykunun her aşamasında geçirdiği süreyi
gösterir. ``Uyanışlar'' özelliği, her öznenin gece boyunca kaç kez
uyandığını kaydeder.

\begin{Shaded}
\begin{Highlighting}[]
\FunctionTok{library}\NormalTok{(tidyverse)}
\FunctionTok{library}\NormalTok{(readxl)}
\FunctionTok{library}\NormalTok{(knitr)}
\FunctionTok{library}\NormalTok{(kableExtra)}
\FunctionTok{library}\NormalTok{(dlookr)}
\FunctionTok{library}\NormalTok{(flextable)}
\FunctionTok{library}\NormalTok{ (ggpubr)}
\FunctionTok{library}\NormalTok{(skimr)}
\FunctionTok{library}\NormalTok{(dlookr)}
\end{Highlighting}
\end{Shaded}

\hypertarget{deux11fiux15fkenlerin-auxe7ux131klamalarux131}{%
\section{Değişkenlerin
Açıklamaları}\label{deux11fiux15fkenlerin-auxe7ux131klamalarux131}}

Yas : Veri setinde bulunan bireylerin yaşlarını belirtir.

Cinsiyet : Veri setinde bulunan bireylerin cinsiyetlerini belirtir.

Yatma\_Vakti : Veri setinde bulunan bireylerin uyudukları saat ve
tarihleri belirtir.

Uyanmak\_Vakti : Veri setinde bulunan nbireylerin yandıkları saat ve
tarihleri belirtir.

Uyku\_Suresi2022 : Veri setinde bulunan bireylerin kaç saat uyuduklarını
belirtir.

Uyku\_Suresi2023 : Veri setinde bulunan bireylerin kaç saat uyuduklarını
belirtir.

REM\_Uyku\_Yuzdesi : Veri setinde bulunan bireylerin REM uykusunda
geçirilen zamanın yüzdesini belirtir.

Derin\_Uyku\_Yuzdesi : Veri setinde bulunan bireylerin Derin uykuda
geçirilen zaman yüzdesini belirtir.

Hafif\_Uyku\_Yuzdesi : Veri setinde bulunan bireylerin Hafif uykuda
geçirilen zaman yüzdesini belirtir.

Uyanıslar : Veri setinde bulunan bireylerin Uykudan kaç kez
kalktıklarını belirtir.

Kafein\_Turekitim : Veri setinde bulunan bireylerin Kafein tüketiminin
mg değerlerini belirtir.

Alkol\_Tutekimi : Veri setinde bulunan bireylerin Alkol tütekimini
haftada ne kadar kullandığını belirtir.

Sigara\_Icme\_Durumu2022 : Veri setinde bulunan bireylerin sigara içme
durumunu ``EVET'' ``HAYIR'' olarak belirtir.

Sigara\_Icme\_Durumu2023 : Veri setinde bulunan bireylerin sigara içme
durumunu ``EVET'' ``HAYIR'' olarak belirtir.

\begin{Shaded}
\begin{Highlighting}[]
\CommentTok{\#|label: tbl{-}Krizantem}
\CommentTok{\#|tbl{-}cap: "Krizantem veri seti"}
\FunctionTok{head}\NormalTok{(Krizantem)}
\end{Highlighting}
\end{Shaded}

\begin{verbatim}
# A tibble: 6 x 19
     ID   Yas Cinsiyet Yatma_Vakti         Uyanma_Vakti        Uyku_su~1 Uyku_~2
  <dbl> <dbl> <chr>    <dttm>              <dttm>                  <dbl> <chr>  
1     1    65 Female   2021-03-06 01:00:00 2021-03-06 07:00:00       6   0.88   
2     3    40 Female   2021-05-25 21:30:00 2021-05-25 05:30:00       8   0.89   
3     4    40 Female   2021-11-03 02:30:00 2021-11-03 08:30:00       6   0.51   
4     6    36 Female   2021-07-01 21:00:00 2021-07-01 04:30:00       7.5 0.9    
5     7    27 Female   2021-07-21 21:00:00 2021-07-21 03:00:00       6   0.54   
6     9    41 Female   2021-04-05 02:30:00 2021-04-05 08:30:00       6   0.79   
# ... with 12 more variables: REM_Uyku_Yuzdesi2022 <dbl>,
#   Derin_Uyku_Yuzdesi <dbl>, Hafif_Uyku_Yuzdesi <dbl>, Uyanıslar <chr>,
#   Kafein_Tuketimi <chr>, Alkol_Tuketimi <chr>, Egzersiz_Sıklıgı <chr>,
#   Uyku_Suresi_2023 <dbl>, Sigara_Icme_Durumu_2022 <chr>,
#   Sigara_Icme_Durumu_2023 <chr>, Degisim <dbl>, REM_Uyku_Yuzdesi2023 <dbl>,
#   and abbreviated variable names 1: Uyku_suresi_2022, 2: Uyku_Verimliligi
\end{verbatim}

\hypertarget{deux11fiux15fkenlerin-tipleri}{%
\section{Değişkenlerin Tipleri}\label{deux11fiux15fkenlerin-tipleri}}

Veri setindeki değişkenlerin istatistiksel analizler yapılması veya
grafiklerin çizilmesi için uygun tipte olması gereklidir. Veri setindeki
değişkenlerin tiplerini str() fonksiyonu ile görebiliriz. Yukarıdaki
listede Krizantem veri setinde yer alan değişken isimleri, tipleri,
gözlem sayıları ve değişkenlere ait bir kaç gözlem verilmektedir.

\begin{Shaded}
\begin{Highlighting}[]
\FunctionTok{str}\NormalTok{(Krizantem)}
\end{Highlighting}
\end{Shaded}

\begin{verbatim}
tibble [452 x 19] (S3: tbl_df/tbl/data.frame)
 $ ID                     : num [1:452] 1 3 4 6 7 9 10 13 15 16 ...
 $ Yas                    : num [1:452] 65 40 40 36 27 41 11 30 36 32 ...
 $ Cinsiyet               : chr [1:452] "Female" "Female" "Female" "Female" ...
 $ Yatma_Vakti            : POSIXct[1:452], format: "2021-03-06 01:00:00" "2021-05-25 21:30:00" ...
 $ Uyanma_Vakti           : POSIXct[1:452], format: "2021-03-06 07:00:00" "2021-05-25 05:30:00" ...
 $ Uyku_suresi_2022       : num [1:452] 6 8 6 7.5 6 6 9 9 8.5 7.5 ...
 $ Uyku_Verimliligi       : chr [1:452] "0.88" "0.89" "0.51" "0.9" ...
 $ REM_Uyku_Yuzdesi2022   : num [1:452] 18 20 23 23 28 28 18 24 20 25 ...
 $ Derin_Uyku_Yuzdesi     : num [1:452] 70 70 25 60 25 55 37 58 32 55 ...
 $ Hafif_Uyku_Yuzdesi     : num [1:452] 12 10 52 17 47 17 45 18 48 20 ...
 $ Uyanıslar              : chr [1:452] "0.0" "1.0" "3.0" "0.0" ...
 $ Kafein_Tuketimi        : chr [1:452] "0.0" "0.0" "50.0" NA ...
 $ Alkol_Tuketimi         : chr [1:452] "0.0" "0.0" "5.0" "0.0" ...
 $ Egzersiz_Sıklıgı       : chr [1:452] "3.0" "3.0" "1.0" "1.0" ...
 $ Uyku_Suresi_2023       : num [1:452] 8 8.5 6 6 6 9 10 10 7 6 ...
 $ Sigara_Icme_Durumu_2022: chr [1:452] "Yes" "Yes" "Yes" "Yes" ...
 $ Sigara_Icme_Durumu_2023: chr [1:452] "No" "No" "No" "No" ...
 $ Degisim                : num [1:452] 2 0.5 0 -1.5 0 3 1 1 -1.5 -1.5 ...
 $ REM_Uyku_Yuzdesi2023   : num [1:452] 16 22 23 24 24 24 25 19 15 19 ...
\end{verbatim}

\hypertarget{hipotez-testleri}{%
\section{Hipotez Testleri}\label{hipotez-testleri}}

Bir bölgede yaşayan insanların 2022 yılı içerisinde ortalama 1 günde en
az 6 saat uyumaktadır.\%5 anlam düzeyinde bu iddia doğru mudur ?

Mu = ortalama uyku zamanı

H0: mu \textless= 6

H1: mu \textgreater{} 6

pdeğeri\textless alfa ise H0 ret, 1\textgreater0.05 olduğundan H0
reddedilmez

\begin{Shaded}
\begin{Highlighting}[]
\FunctionTok{t.test}\NormalTok{(}\AttributeTok{x=}\NormalTok{Krizantem}\SpecialCharTok{$}\NormalTok{Uyku\_suresi\_2022 , }\AttributeTok{alternative =} \StringTok{"l"}\NormalTok{, }\AttributeTok{mu=}\DecValTok{6}\NormalTok{, }\AttributeTok{conf.level =} \FloatTok{0.95}\NormalTok{)}
\end{Highlighting}
\end{Shaded}

\begin{verbatim}

    One Sample t-test

data:  Krizantem$Uyku_suresi_2022
t = 35.957, df = 451, p-value = 1
alternative hypothesis: true mean is less than 6
95 percent confidence interval:
     -Inf 7.532894
sample estimates:
mean of x 
 7.465708 
\end{verbatim}

\begin{Shaded}
\begin{Highlighting}[]
\FunctionTok{ggplot}\NormalTok{(Krizantem, }\FunctionTok{aes}\NormalTok{(}\AttributeTok{x =}\NormalTok{ Yas))}\SpecialCharTok{+}
\FunctionTok{geom\_bar}\NormalTok{(}\AttributeTok{fill=}\StringTok{"darkolivegreen3"}\NormalTok{, }\AttributeTok{colour=}\StringTok{"black"}\NormalTok{)}\SpecialCharTok{+}
\FunctionTok{geom\_text}\NormalTok{(}\AttributeTok{stat =} \StringTok{"count"}\NormalTok{, }\FunctionTok{aes}\NormalTok{(}\AttributeTok{label=}\NormalTok{..count..),}\AttributeTok{hjust=}\SpecialCharTok{{-}}\FloatTok{0.4}\NormalTok{, }\AttributeTok{size =}\DecValTok{3}\NormalTok{)}\SpecialCharTok{+}
\FunctionTok{theme}\NormalTok{(}\AttributeTok{plot.background =} \FunctionTok{element\_rect}\NormalTok{(}\AttributeTok{fill =}\StringTok{"rosybrown2"}\NormalTok{), }\AttributeTok{axis.text.x =} \FunctionTok{element\_text}\NormalTok{(}\AttributeTok{size =} \DecValTok{8}\NormalTok{))}\SpecialCharTok{+}
\FunctionTok{coord\_flip}\NormalTok{()}\SpecialCharTok{+}
\FunctionTok{labs}\NormalTok{(}\AttributeTok{title =} \StringTok{"YAŞ"}\NormalTok{,}
     \AttributeTok{x =} \ConstantTok{NULL}\NormalTok{,}
     \AttributeTok{y =} \ConstantTok{NULL}\NormalTok{)}
\end{Highlighting}
\end{Shaded}

\begin{figure}[H]

{\centering \includegraphics{Krizantem_files/figure-pdf/unnamed-chunk-6-1.pdf}

}

\end{figure}

Bir grup denekler üzerinde uyku verimliliği incelenen insanların
ortalama yaşlarının 28'den fazla olduğu iddia edilmektedir.Bu amaçla 451
kişi test edilmiş ve veri setinde sonuçlar verilmiştir. Veriler normal
dağılıma sahip olduğunu varsayarak \%5 anlam düzeyi için bu iddia doğru
mudur ?

Mu = ortalama yaş

H0: mu = Popülasyonun yaş ortalaması 28'dir.

H1: mu = Popülasyonun yaş ortalaması 28'den farklıdır.

pdeğeri\textless alfa ise H0 ret, 2.2e-16\textless0.05 olduğundan H0
reddedilir ve H1 hipotezi doğru kabul edilir.

\begin{Shaded}
\begin{Highlighting}[]
\FunctionTok{t.test}\NormalTok{(}\AttributeTok{x=}\NormalTok{Krizantem}\SpecialCharTok{$}\NormalTok{Yas, }\AttributeTok{mu=}\DecValTok{28}\NormalTok{, }\AttributeTok{alternative =} \StringTok{"g"}\NormalTok{, }\AttributeTok{conf.level =} \FloatTok{0.95}\NormalTok{)}
\end{Highlighting}
\end{Shaded}

\begin{verbatim}

    One Sample t-test

data:  Krizantem$Yas
t = 19.829, df = 451, p-value < 2.2e-16
alternative hypothesis: true mean is greater than 28
95 percent confidence interval:
 39.2642     Inf
sample estimates:
mean of x 
  40.2854 
\end{verbatim}

\hypertarget{eux15fli-guxf6zlemler-iuxe7in-hipotez-testi}{%
\section{Eşli Gözlemler İçin Hipotez
Testi}\label{eux15fli-guxf6zlemler-iuxe7in-hipotez-testi}}

Sigara tüketiminin gittikçe arttığı gözlemlenen dünyamızda bu durumun
uyku düzenine etkilerinin olduğunu inceleyen bir grup bilim insanı
sigara tüketenlerin sigara tüketimini bıraktıktan sonra uyku süresinin
arttığını iddia etmektedir.. Bu iddiayı test etmek için bir bölgedeki
erkek ve kadınların son iki yılın uyku süresi karşılaştırılmıştır. \%5
anlam düzeyinde bu iddiayı~test~ediniz

H0: Mu = 0

H1: Mu \textgreater{} 0

\begin{Shaded}
\begin{Highlighting}[]
\FunctionTok{t.test}\NormalTok{(}\AttributeTok{x =}\NormalTok{Krizantem}\SpecialCharTok{$}\NormalTok{Uyku\_suresi\_2022, }\AttributeTok{y =}\NormalTok{ Krizantem}\SpecialCharTok{$}\NormalTok{Uyku\_Suresi\_2023, }\AttributeTok{mu =} \DecValTok{0}\NormalTok{ , }\AttributeTok{alternative =} \StringTok{"g"}\NormalTok{, }\AttributeTok{conf.level =} \FloatTok{0.95}\NormalTok{, }\AttributeTok{paired =} \ConstantTok{TRUE}\NormalTok{ )}
\end{Highlighting}
\end{Shaded}

\begin{verbatim}

    Paired t-test

data:  Krizantem$Uyku_suresi_2022 and Krizantem$Uyku_Suresi_2023
t = -6.896, df = 451, p-value = 1
alternative hypothesis: true mean difference is greater than 0
95 percent confidence interval:
 -0.6578841        Inf
sample estimates:
mean difference 
     -0.5309735 
\end{verbatim}

\hypertarget{normallik-testi}{%
\section{Normallik Testi}\label{normallik-testi}}

Krizantem veri seti içerisinde Yas değişkeninin normal dağılımını
histogram grafiği ile belirtiniz

\begin{Shaded}
\begin{Highlighting}[]
\FunctionTok{library}\NormalTok{(tidyverse)}
\FunctionTok{hist}\NormalTok{(}\AttributeTok{x =}\NormalTok{ Krizantem}\SpecialCharTok{$}\NormalTok{Yas, }
     \AttributeTok{main=} \StringTok{"Yaş"}\NormalTok{,}
     \AttributeTok{border =} \StringTok{"Gray11"}\NormalTok{,}
     \AttributeTok{col =} \StringTok{"SpringGreen2"}\NormalTok{,}
     \AttributeTok{freq =} \ConstantTok{FALSE}\NormalTok{)}\SpecialCharTok{+}
  \FunctionTok{ggplot}\NormalTok{(Krizantem,}\FunctionTok{aes}\NormalTok{(}\AttributeTok{x=}\NormalTok{Yas))}\SpecialCharTok{+}
  \FunctionTok{geom\_histogram}\NormalTok{(}\FunctionTok{aes}\NormalTok{(}\AttributeTok{y =} \FunctionTok{after\_stat}\NormalTok{(density)))}\SpecialCharTok{+}
  \FunctionTok{geom\_function}\NormalTok{(}\AttributeTok{fun =}\NormalTok{dnorm, }\AttributeTok{args=}\FunctionTok{list}\NormalTok{(}\AttributeTok{mean=}\FunctionTok{mean}\NormalTok{(Krizantem}\SpecialCharTok{$}\NormalTok{Yas),}\AttributeTok{sd=}\FunctionTok{sd}\NormalTok{(Krizantem}\SpecialCharTok{$}\NormalTok{Yas)),}\AttributeTok{linewidth=}\DecValTok{2}\NormalTok{)}
\end{Highlighting}
\end{Shaded}

\begin{figure}[H]

{\centering \includegraphics{Krizantem_files/figure-pdf/unnamed-chunk-9-1.pdf}

}

\end{figure}

\begin{verbatim}
NULL
\end{verbatim}

\hypertarget{q---q-grafiux11fi}{%
\section{Q - Q Grafiği}\label{q---q-grafiux11fi}}

\hypertarget{shapiro-wilk-normallik-testi}{%
\section{Shapiro-Wilk Normallik
Testi}\label{shapiro-wilk-normallik-testi}}

REM uyku yüzdesi 2022 için Shapiro-Wilk Normallik Testi

H0: Verilerin içerisinde bulunan REM uyku yüzdesi değişkeni normal
dağılıma sahiptir.

H1: Verilerin içerisinde bulunan REM uyku yüzdesi normal dağılıma sahip
değildir.

Karar Kuralı: p-değeri\textless alfa ise H0 red,
p-değeri\textgreater=alfa ise H0 reddedilemez.

\begin{Shaded}
\begin{Highlighting}[]
\FunctionTok{shapiro.test}\NormalTok{(}\AttributeTok{x =}\NormalTok{Krizantem}\SpecialCharTok{$}\NormalTok{REM\_Uyku\_Yuzdesi2022)}
\end{Highlighting}
\end{Shaded}

\begin{verbatim}

    Shapiro-Wilk normality test

data:  Krizantem$REM_Uyku_Yuzdesi2022
W = 0.95483, p-value = 1.549e-10
\end{verbatim}

\begin{Shaded}
\begin{Highlighting}[]
\FunctionTok{library}\NormalTok{(tidyverse)}
\FunctionTok{hist}\NormalTok{(}\AttributeTok{x =}\NormalTok{ Krizantem}\SpecialCharTok{$}\NormalTok{REM\_Uyku\_Yuzdesi2022, }
     \AttributeTok{main=} \StringTok{""}\NormalTok{,}
     \AttributeTok{border =} \StringTok{"Gray11"}\NormalTok{,}
     \AttributeTok{col =} \StringTok{"SpringGreen2"}\NormalTok{,}
     \AttributeTok{freq =} \ConstantTok{FALSE}\NormalTok{)}\SpecialCharTok{+}
  \FunctionTok{ggplot}\NormalTok{(Krizantem,}\FunctionTok{aes}\NormalTok{(}\AttributeTok{x=}\NormalTok{REM\_Uyku\_Yuzdesi2022))}\SpecialCharTok{+}
  \FunctionTok{geom\_histogram}\NormalTok{(}\FunctionTok{aes}\NormalTok{(}\AttributeTok{y =} \FunctionTok{after\_stat}\NormalTok{(density)))}\SpecialCharTok{+}
  \FunctionTok{geom\_function}\NormalTok{(}\AttributeTok{fun =}\NormalTok{dnorm, }\AttributeTok{args=}\FunctionTok{list}\NormalTok{(}\AttributeTok{mean=}\FunctionTok{mean}\NormalTok{(Krizantem}\SpecialCharTok{$}\NormalTok{REM\_Uyku\_Yuzdesi2022),}\AttributeTok{sd=}\FunctionTok{sd}\NormalTok{(Krizantem}\SpecialCharTok{$}\NormalTok{REM\_Uyku\_Yuzdesi2022)),}\AttributeTok{linewidth=}\DecValTok{2}\NormalTok{)}
\end{Highlighting}
\end{Shaded}

\begin{figure}[H]

{\centering \includegraphics{Krizantem_files/figure-pdf/unnamed-chunk-11-1.pdf}

}

\end{figure}

\begin{verbatim}
NULL
\end{verbatim}

\begin{Shaded}
\begin{Highlighting}[]
\FunctionTok{sort}\NormalTok{(Krizantem}\SpecialCharTok{$}\NormalTok{REM\_Uyku\_Yuzdesi2022)}
\end{Highlighting}
\end{Shaded}

\begin{verbatim}
  [1] 15 15 15 15 15 15 15 15 15 15 15 15 15 15 18 18 18 18 18 18 18 18 18 18 18
 [26] 18 18 18 18 18 18 18 18 18 18 18 18 18 18 18 18 18 18 18 18 18 18 18 18 18
 [51] 18 18 18 18 18 18 18 18 18 18 18 18 18 19 19 19 19 19 19 19 19 19 19 19 20
 [76] 20 20 20 20 20 20 20 20 20 20 20 20 20 20 20 20 20 20 20 20 20 20 20 20 20
[101] 20 20 20 20 20 20 20 20 20 20 20 20 20 20 20 20 20 20 20 20 20 20 20 20 20
[126] 20 20 20 20 20 20 20 20 20 20 20 20 20 20 20 20 20 20 20 20 20 20 20 20 20
[151] 20 20 20 20 20 20 20 20 20 20 20 20 20 20 20 20 21 22 22 22 22 22 22 22 22
[176] 22 22 22 22 22 22 22 22 22 22 22 22 22 22 22 22 22 22 22 22 22 22 22 22 22
[201] 22 22 22 22 22 22 22 22 22 22 22 22 22 22 22 22 22 22 22 22 22 22 22 22 22
[226] 22 22 22 22 22 22 22 22 22 23 23 23 23 23 23 23 23 23 23 23 23 23 23 23 23
[251] 23 23 23 23 23 23 23 23 23 23 23 23 23 23 23 23 23 23 23 23 23 23 23 23 23
[276] 23 23 23 23 23 23 23 23 23 23 23 23 23 23 23 24 24 24 24 24 24 24 24 24 24
[301] 24 24 24 24 24 24 24 24 24 24 24 24 24 24 24 24 25 25 25 25 25 25 25 25 25
[326] 25 25 25 25 25 25 25 25 25 25 25 25 25 25 25 25 25 25 25 25 25 25 26 26 26
[351] 26 26 26 26 26 26 26 26 26 26 26 26 27 27 27 27 27 27 27 27 27 27 27 27 27
[376] 27 27 27 27 27 27 27 27 27 27 27 27 28 28 28 28 28 28 28 28 28 28 28 28 28
[401] 28 28 28 28 28 28 28 28 28 28 28 28 28 28 28 28 28 28 28 28 28 28 28 28 28
[426] 28 28 28 28 28 28 28 28 28 28 28 28 28 28 28 28 28 28 28 28 30 30 30 30 30
[451] 30 30
\end{verbatim}

\begin{Shaded}
\begin{Highlighting}[]
\FunctionTok{ggplot}\NormalTok{(}\AttributeTok{data =}\NormalTok{ Krizantem, }\FunctionTok{aes}\NormalTok{(}\AttributeTok{sample =}\NormalTok{ REM\_Uyku\_Yuzdesi2022))}\SpecialCharTok{+}
  \FunctionTok{stat\_qq}\NormalTok{(}\AttributeTok{distribution =}\NormalTok{ qnorm )}\SpecialCharTok{+}
  \FunctionTok{stat\_qq\_line}\NormalTok{(}\AttributeTok{distribution =}\NormalTok{ qnorm)}
\end{Highlighting}
\end{Shaded}

\begin{figure}[H]

{\centering \includegraphics{Krizantem_files/figure-pdf/unnamed-chunk-13-1.pdf}

}

\end{figure}

REM uyku yüzdesi 2023 için Shapiro-Wilk Normallik Testi

H0: Verilerin içerisinde bulunan REM uyku yüzdesi değişkeni normal
dağılıma sahiptir.

H1: Verilerin içerisinde bulunan REM uyku yüzdesi normal dağılıma sahip
değildir.

Karar Kuralı: p-değeri\textless alfa ise H0 red,
p-değeri\textgreater=alfa ise H0 reddedilemez.

\begin{Shaded}
\begin{Highlighting}[]
\FunctionTok{shapiro.test}\NormalTok{(}\AttributeTok{x =}\NormalTok{Krizantem}\SpecialCharTok{$}\NormalTok{REM\_Uyku\_Yuzdesi2023)}
\end{Highlighting}
\end{Shaded}

\begin{verbatim}

    Shapiro-Wilk normality test

data:  Krizantem$REM_Uyku_Yuzdesi2023
W = 0.93921, p-value = 1.24e-12
\end{verbatim}

\begin{Shaded}
\begin{Highlighting}[]
\FunctionTok{library}\NormalTok{(tidyverse)}
\FunctionTok{hist}\NormalTok{(}\AttributeTok{x =}\NormalTok{ Krizantem}\SpecialCharTok{$}\NormalTok{REM\_Uyku\_Yuzdesi2023, }
     \AttributeTok{main=} \StringTok{""}\NormalTok{,}
     \AttributeTok{border =} \StringTok{"Gray11"}\NormalTok{,}
     \AttributeTok{col =} \StringTok{"SpringGreen2"}\NormalTok{,}
     \AttributeTok{freq =} \ConstantTok{FALSE}\NormalTok{)}\SpecialCharTok{+}
  \FunctionTok{ggplot}\NormalTok{(Krizantem,}\FunctionTok{aes}\NormalTok{(}\AttributeTok{x=}\NormalTok{REM\_Uyku\_Yuzdesi2023))}\SpecialCharTok{+}
  \FunctionTok{geom\_histogram}\NormalTok{(}\FunctionTok{aes}\NormalTok{(}\AttributeTok{y =} \FunctionTok{after\_stat}\NormalTok{(density)))}\SpecialCharTok{+}
  \FunctionTok{geom\_function}\NormalTok{(}\AttributeTok{fun =}\NormalTok{dnorm, }\AttributeTok{args=}\FunctionTok{list}\NormalTok{(}\AttributeTok{mean=}\FunctionTok{mean}\NormalTok{(Krizantem}\SpecialCharTok{$}\NormalTok{REM\_Uyku\_Yuzdesi2023),}\AttributeTok{sd=}\FunctionTok{sd}\NormalTok{(Krizantem}\SpecialCharTok{$}\NormalTok{REM\_Uyku\_Yuzdesi2023)),}\AttributeTok{linewidth=}\DecValTok{2}\NormalTok{)}
\end{Highlighting}
\end{Shaded}

\begin{figure}[H]

{\centering \includegraphics{Krizantem_files/figure-pdf/unnamed-chunk-15-1.pdf}

}

\end{figure}

\begin{verbatim}
NULL
\end{verbatim}

\begin{Shaded}
\begin{Highlighting}[]
\FunctionTok{sort}\NormalTok{(Krizantem}\SpecialCharTok{$}\NormalTok{REM\_Uyku\_Yuzdesi2023)}
\end{Highlighting}
\end{Shaded}

\begin{verbatim}
  [1] 15 15 15 15 15 15 15 15 15 15 15 15 15 15 15 15 15 15 15 15 15 15 15 15 15
 [26] 15 15 15 15 15 16 16 16 16 16 16 16 16 16 16 16 16 16 16 16 16 16 16 16 16
 [51] 16 16 16 16 16 16 16 17 17 17 17 17 17 17 17 17 17 17 17 17 17 17 17 17 17
 [76] 17 17 17 17 17 17 17 17 17 17 17 17 17 17 17 17 17 17 17 18 18 18 18 18 18
[101] 18 18 18 18 18 18 18 18 18 18 18 18 18 18 18 18 18 18 18 18 18 18 18 19 19
[126] 19 19 19 19 19 19 19 19 19 19 19 19 19 19 19 19 19 19 19 19 19 19 19 19 19
[151] 19 19 19 20 20 20 20 20 20 20 20 20 20 20 20 20 20 20 20 20 20 20 20 20 20
[176] 20 20 20 20 20 20 20 20 21 21 21 21 21 21 21 21 21 21 21 21 21 21 21 21 21
[201] 21 21 21 21 21 21 21 21 22 22 22 22 22 22 22 22 22 22 22 22 22 22 22 22 22
[226] 22 22 22 22 22 22 22 22 22 23 23 23 23 23 23 23 23 23 23 23 23 23 23 23 23
[251] 23 23 23 23 23 23 23 23 23 23 23 23 23 23 24 24 24 24 24 24 24 24 24 24 24
[276] 24 24 24 24 24 24 24 24 24 24 24 24 24 24 24 24 24 25 25 25 25 25 25 25 25
[301] 25 25 25 25 25 25 25 25 25 25 25 25 25 25 25 25 25 25 25 25 25 25 26 26 26
[326] 26 26 26 26 26 26 26 26 26 26 26 26 26 26 26 26 26 26 26 26 26 26 26 26 26
[351] 26 26 26 26 26 26 26 26 27 27 27 27 27 27 27 27 27 27 27 27 27 27 27 27 27
[376] 27 27 27 27 27 27 27 27 27 27 27 27 27 27 28 28 28 28 28 28 28 28 28 28 28
[401] 28 28 28 28 28 28 28 28 28 28 28 28 28 28 28 28 28 28 28 28 28 28 29 29 29
[426] 29 29 29 29 29 29 29 29 29 29 29 29 29 29 29 29 29 29 29 29 29 29 29 29 29
[451] 29 29
\end{verbatim}

\begin{Shaded}
\begin{Highlighting}[]
\FunctionTok{ggplot}\NormalTok{(}\AttributeTok{data =}\NormalTok{ Krizantem, }\FunctionTok{aes}\NormalTok{(}\AttributeTok{sample =}\NormalTok{ REM\_Uyku\_Yuzdesi2023))}\SpecialCharTok{+}
  \FunctionTok{stat\_qq}\NormalTok{(}\AttributeTok{distribution =}\NormalTok{ qnorm )}\SpecialCharTok{+}
  \FunctionTok{stat\_qq\_line}\NormalTok{(}\AttributeTok{distribution =}\NormalTok{ qnorm)}
\end{Highlighting}
\end{Shaded}

\begin{figure}[H]

{\centering \includegraphics{Krizantem_files/figure-pdf/unnamed-chunk-17-1.pdf}

}

\end{figure}

Şimdi verileri normal dağılıma varsayarak hipotez testine devam edelim.
H1 : M1!=M2 oldğundan t-testi aşağıdaki gibi yapılır.

\begin{Shaded}
\begin{Highlighting}[]
\FunctionTok{t.test}\NormalTok{(}\AttributeTok{x =}\NormalTok{ Krizantem}\SpecialCharTok{$}\NormalTok{REM\_Uyku\_Yuzdesi2022, }\AttributeTok{y =}\NormalTok{ Krizantem}\SpecialCharTok{$}\NormalTok{REM\_Uyku\_Yuzdesi2023, }\AttributeTok{mu =} \DecValTok{0}\NormalTok{ ,}\AttributeTok{alternative =}\StringTok{"t"}\NormalTok{, }\AttributeTok{var.equal =} \ConstantTok{TRUE}\NormalTok{)}
\end{Highlighting}
\end{Shaded}

\begin{verbatim}

    Two Sample t-test

data:  Krizantem$REM_Uyku_Yuzdesi2022 and Krizantem$REM_Uyku_Yuzdesi2023
t = 2.0319, df = 902, p-value = 0.04246
alternative hypothesis: true difference in means is not equal to 0
95 percent confidence interval:
 0.0183258 1.0568954
sample estimates:
mean of x mean of y 
 22.61504  22.07743 
\end{verbatim}

\begin{Shaded}
\begin{Highlighting}[]
\FunctionTok{t.test}\NormalTok{(}\AttributeTok{x =}\NormalTok{ Krizantem}\SpecialCharTok{$}\NormalTok{REM\_Uyku\_Yuzdesi2022, }\AttributeTok{y =}\NormalTok{ Krizantem}\SpecialCharTok{$}\NormalTok{REM\_Uyku\_Yuzdesi2023, }\AttributeTok{mu =} \DecValTok{0}\NormalTok{ ,}\AttributeTok{alternative =}\StringTok{"t"}\NormalTok{, }\AttributeTok{var.equal =} \ConstantTok{FALSE}\NormalTok{)}
\end{Highlighting}
\end{Shaded}

\begin{verbatim}

    Welch Two Sample t-test

data:  Krizantem$REM_Uyku_Yuzdesi2022 and Krizantem$REM_Uyku_Yuzdesi2023
t = 2.0319, df = 862.42, p-value = 0.04247
alternative hypothesis: true difference in means is not equal to 0
95 percent confidence interval:
 0.01829378 1.05692746
sample estimates:
mean of x mean of y 
 22.61504  22.07743 
\end{verbatim}

Varyanslar eşit ve eşit değil durumları için p-değeri \%5'ten küçük
olduğundan HO hipotezi reddedilir. Yani \%5 anlam düzeyinde REM uyku
düzeylerinde değişim olmuştur.

\hypertarget{regresyon-analizi}{%
\section{Regresyon Analizi}\label{regresyon-analizi}}

\begin{Shaded}
\begin{Highlighting}[]
\FunctionTok{library}\NormalTok{(ggplot2)}
\FunctionTok{ggplot}\NormalTok{(}\AttributeTok{data =}\NormalTok{ Krizantem, }\FunctionTok{aes}\NormalTok{(}\AttributeTok{x =}\NormalTok{ Derin\_Uyku\_Yuzdesi, }\AttributeTok{y =}\NormalTok{Uyku\_Suresi\_2023))}\SpecialCharTok{+}
  \FunctionTok{geom\_point}\NormalTok{(}\AttributeTok{size =} \DecValTok{3}\NormalTok{)}\SpecialCharTok{+}
  \FunctionTok{geom\_smooth}\NormalTok{(}\AttributeTok{method =} \StringTok{"lm"}\NormalTok{, }\AttributeTok{se =} \ConstantTok{FALSE}\NormalTok{ , }\AttributeTok{color=}\StringTok{"Red"}\NormalTok{, }\AttributeTok{linewidth =}\DecValTok{1}\NormalTok{ )}
\end{Highlighting}
\end{Shaded}

\begin{verbatim}
`geom_smooth()` using formula = 'y ~ x'
\end{verbatim}

\begin{figure}[H]

{\centering \includegraphics{Krizantem_files/figure-pdf/unnamed-chunk-20-1.pdf}

}

\end{figure}

\begin{Shaded}
\begin{Highlighting}[]
\NormalTok{model }\OtherTok{=} \FunctionTok{lm}\NormalTok{(Uyku\_Suresi\_2023}\SpecialCharTok{\textasciitilde{}}\NormalTok{Derin\_Uyku\_Yuzdesi, }\AttributeTok{data =}\NormalTok{ Krizantem)}
\NormalTok{model}
\end{Highlighting}
\end{Shaded}

\begin{verbatim}

Call:
lm(formula = Uyku_Suresi_2023 ~ Derin_Uyku_Yuzdesi, data = Krizantem)

Coefficients:
       (Intercept)  Derin_Uyku_Yuzdesi  
         7.9917898           0.0000926  
\end{verbatim}

\begin{Shaded}
\begin{Highlighting}[]
\NormalTok{hatalar }\OtherTok{=} \FunctionTok{residuals}\NormalTok{(model)}
\FunctionTok{head}\NormalTok{(hatalar)}
\end{Highlighting}
\end{Shaded}

\begin{verbatim}
           1            2            3            4            5            6 
 0.001727936  0.501727936 -1.994104911 -1.997346030 -1.994104911  1.003116987 
\end{verbatim}

\begin{Shaded}
\begin{Highlighting}[]
\FunctionTok{qqnorm}\NormalTok{(hatalar)}
\FunctionTok{qqline}\NormalTok{(hatalar)}
\end{Highlighting}
\end{Shaded}

\begin{figure}[H]

{\centering \includegraphics{Krizantem_files/figure-pdf/unnamed-chunk-20-2.pdf}

}

\end{figure}

\begin{Shaded}
\begin{Highlighting}[]
\FunctionTok{shapiro.test}\NormalTok{(hatalar)}
\end{Highlighting}
\end{Shaded}

\begin{verbatim}

    Shapiro-Wilk normality test

data:  hatalar
W = 0.8875, p-value < 2.2e-16
\end{verbatim}

\hypertarget{baux11fux131msux131zlux131k-testi}{%
\section{Bağımsızlık Testi}\label{baux11fux131msux131zlux131k-testi}}

\begin{Shaded}
\begin{Highlighting}[]
\NormalTok{tablo }\OtherTok{=} \FunctionTok{rbind}\NormalTok{(Krizantem}\SpecialCharTok{$}\NormalTok{Yas,Krizantem}\SpecialCharTok{$}\NormalTok{Derin\_Uyku\_Yuzdesi)}
\FunctionTok{chisq.test}\NormalTok{(tablo)}
\end{Highlighting}
\end{Shaded}

\begin{verbatim}

    Pearson's Chi-squared test

data:  tablo
X-squared = 2156.6, df = 451, p-value < 2.2e-16
\end{verbatim}

Ho : Bağımsızdır

H1 : Bağımlıdır

P değeri 0'a çok yakın bir değer çıktığı için H0 hipotezi reddediliyor
yani Veri setinde bulunan ``Yaş'' ve ``Derin uyku yüzdesi'' birbiriyle
ilişkilidir.

\hypertarget{varyans-analizi}{%
\section{Varyans Analizi}\label{varyans-analizi}}

\begin{Shaded}
\begin{Highlighting}[]
\NormalTok{Krizantem}\SpecialCharTok{\%\textgreater{}\%}\FunctionTok{group\_by}\NormalTok{(Cinsiyet)}\SpecialCharTok{\%\textgreater{}\%}\FunctionTok{summarise}\NormalTok{(}\AttributeTok{Ortalama=} \FunctionTok{mean}\NormalTok{(Hafif\_Uyku\_Yuzdesi))}
\end{Highlighting}
\end{Shaded}

\begin{verbatim}
# A tibble: 2 x 2
  Cinsiyet Ortalama
  <chr>       <dbl>
1 Female       25.2
2 Male         23.9
\end{verbatim}

\begin{Shaded}
\begin{Highlighting}[]
\FunctionTok{ggplot}\NormalTok{(}\AttributeTok{data =}\NormalTok{ Krizantem, }\FunctionTok{aes}\NormalTok{(}\AttributeTok{x =}\NormalTok{ Cinsiyet, }\AttributeTok{y=}\NormalTok{Hafif\_Uyku\_Yuzdesi, }\AttributeTok{fill =}\NormalTok{Cinsiyet))}\SpecialCharTok{+}
  \FunctionTok{geom\_boxplot}\NormalTok{()}
\end{Highlighting}
\end{Shaded}

\begin{figure}[H]

{\centering \includegraphics{Krizantem_files/figure-pdf/unnamed-chunk-22-1.pdf}

}

\end{figure}

\begin{Shaded}
\begin{Highlighting}[]
\NormalTok{Female }\OtherTok{=}\NormalTok{ Krizantem}\SpecialCharTok{$}\NormalTok{Hafif\_Uyku\_Yuzdesi[Krizantem}\SpecialCharTok{$}\NormalTok{Cinsiyet}\SpecialCharTok{==}\StringTok{"Female"}\NormalTok{]}
\NormalTok{Male }\OtherTok{=}\NormalTok{ Krizantem}\SpecialCharTok{$}\NormalTok{Hafif\_Uyku\_Yuzdesi[Krizantem}\SpecialCharTok{$}\NormalTok{Cinsiyet}\SpecialCharTok{==}\StringTok{"Male"}\NormalTok{]}
\FunctionTok{mean}\NormalTok{(Female)}
\end{Highlighting}
\end{Shaded}

\begin{verbatim}
[1] 25.1875
\end{verbatim}

\begin{Shaded}
\begin{Highlighting}[]
\FunctionTok{mean}\NormalTok{(Male)}
\end{Highlighting}
\end{Shaded}

\begin{verbatim}
[1] 23.94737
\end{verbatim}

\begin{Shaded}
\begin{Highlighting}[]
\CommentTok{\# Normallik Testi}
\FunctionTok{shapiro.test}\NormalTok{(Female)}
\end{Highlighting}
\end{Shaded}

\begin{verbatim}

    Shapiro-Wilk normality test

data:  Female
W = 0.74467, p-value < 2.2e-16
\end{verbatim}

\begin{Shaded}
\begin{Highlighting}[]
\FunctionTok{shapiro.test}\NormalTok{(Male)}
\end{Highlighting}
\end{Shaded}

\begin{verbatim}

    Shapiro-Wilk normality test

data:  Male
W = 0.74617, p-value < 2.2e-16
\end{verbatim}

\begin{Shaded}
\begin{Highlighting}[]
\CommentTok{\#Varyansların Eşitlik Testi}
\FunctionTok{var.test}\NormalTok{(Female,Male)}
\end{Highlighting}
\end{Shaded}

\begin{verbatim}

    F test to compare two variances

data:  Female and Male
F = 1.1848, num df = 223, denom df = 227, p-value = 0.204
alternative hypothesis: true ratio of variances is not equal to 1
95 percent confidence interval:
 0.9118571 1.5399232
sample estimates:
ratio of variances 
          1.184802 
\end{verbatim}

\begin{Shaded}
\begin{Highlighting}[]
\NormalTok{test }\OtherTok{=} \FunctionTok{aov}\NormalTok{(}\AttributeTok{formula =}\NormalTok{ Hafif\_Uyku\_Yuzdesi}\SpecialCharTok{\textasciitilde{}}\NormalTok{Cinsiyet, }\AttributeTok{data =}\NormalTok{Krizantem)}
\FunctionTok{summary}\NormalTok{(test)}
\end{Highlighting}
\end{Shaded}

\begin{verbatim}
             Df Sum Sq Mean Sq F value Pr(>F)
Cinsiyet      1    174   173.8   0.741   0.39
Residuals   450 105589   234.6               
\end{verbatim}

Cinsiyetlere bağlı olarak ortalama hafif uyku yüzdesi değişmektemidir ?

H0 : Mu1=Mu2

H1 : Ortalama farklı

P değeri 0.39 çıktığından dolayı 0'a ne kadar çok yakınsa H0 hipotezinin
yanlışlığı doğrulanır. Sonuç olarak H0 hipotezi rededilmiştir.



\end{document}
